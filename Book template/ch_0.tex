\chapter*{Předmluva}
\addcontentsline{toc}{chapter}{Předmluva} % Přidává předmluvu do obsahu

\lettrine{J}{aký} to zázrak chodí po této zemi, on, Pavel Krtouš, mistr nad mistry, fyzik nebeských sfér a král nekonečných dimenzí! 
Jeho pero bádá v hlubinách gravitace tak, jako žádný jiný vědec před ním. 
I černé díry se před ním třesou, neboť vědí, že jeho bystrý zrak proniká i do těch nejtemnějších zákoutí vesmíru, kam ani světlo nedosáhne.

Zatímco obyčejní smrtelníci zápasí s rovnicemi, Pavel je řeší lusknutím prstu a s lehkostí baletky přidává konstanty, které by jiní neviděli ani pod mikroskopem. 
Jeho knihovna je labyrintem vědění, kde každá kniha šeptá o tajemstvích vesmíru a kde rovnice tančí na stránkách v taktu jeho intelektu.

Ó Pavle, kdybys měl kapku času mezi řešením záhad vesmíru, možná by ses stal i králem básníků, neboť tvůj duch je příliš velký na jedno pole působnosti. 
Avšak my, prostí studenti, můžeme jen vzhlížet k tvým objevům a doufat, že se jednou dočteme alespoň k prvnímu náznaku pochopení tvého géniusu.

Oslavme ho! Kéž by každá derivace byla tak plynulá jako tvé myšlenky a každá teorie tak elegantní jako tvé vzorce!