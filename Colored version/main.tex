\documentclass[colored,czech,online]{CBTGTR}
\usepackage{F10}

\begin{document}
\chapter{Ukázka prostředí}
Všechna prostředí jsou:
\begin{table*}[h]
\begin{tabular}{| r | c | p{6cm} |}
	\hline
	Lemmátko & \texttt{$\backslash$begin\{lemmish\}[volitelný název]} & Triviální pomocný výsledek, nedokazujeme, ponecháno k rozmyšlení.\\
	\hline
	Lemma & \texttt{$\backslash$begin\{lemma\}[volitelný název]} & Netriviální a často pomocný výsledek, nejedná se důležitostí o větu, ale je dokázán.\\
	\hline
	Věta & \texttt{$\backslash$begin\{theorem\}[volitelný název]} & Podstatný výsledek textu, dokazujeme a pokládáme důraz na důsledky.\\
	\hline
	Tvrzení & \texttt{$\backslash$begin\{claim\}[volitelný název]} & Potřebujete dropnout Poincarého lemma, ale nechcete zajebat 20 stran důkazem? Od toho je tu tvrzení. Vždy ale odkázat na zdroj, kde lze důkaz najít!\\
	\hline
	Definice & \texttt{$\backslash$begin\{definition\}[volitelný název]} & Definice...\\
	\hline
	Důkaz & \texttt{$\backslash$begin\{proof\}[nějaké věty]} & Prostředí začíná \textit{Důkaz \uv{nějaké věty}:} kde \uv{nějaké věty} může být prázdné.\\
	\hline
\end{tabular}
\end{table*}
\section{Demonstrace prostředí}
Níže je definováno lemmátko bez názvu. 
Název by samozřejmě mohl být definován.
Všimněte si taky, že každou větu v kódu začínám na novém řádku.
Ve vygenerovaném textu se jeden linebreak projeví jako mezera, takže text je v pořádku.
Nicméně bude se nám pak nohem lépe v GitHubu sledovat změny. 
Psát každou větu na vlastní řádek se prostě mega vyplácí...

Nyní jsme do kódu vložili prázdný řádek. 
Tak začíná nový odstavec.
Odstavec \hl{NEZAČÍNÁME} tak, že vložíme \texttt{$\backslash\backslash$}, ale prázdným řádkem.
A teď už ta prostředí:
\begin{lemmish}
Pro všechna $a,b\in\mathbb{R}$ platí, že
\begin{equation}
	(a+b)^2=a^2+b^2 \iff a=0 \vee b=0
\end{equation}
\end{lemmish}
Další prostředí je pak Lemma.
\begin{lemma}[Červeňanovo blbé lemma]
	Pro všechna $a,b\in\mathbb{R}$ platí, že
	\begin{equation}
		(a+b)^2=a^2+b^2 \iff a=0 \vee b=0
	\end{equation}
\end{lemma}
% Zde by se mělo zmíit, kdo následující důkaz texal, abychom to nemuseli hledat na Gitu. Takto:
% Sepsal:
%	Jan, 25.9.2024
% Opravy:
%	Petr, 10.11.2024
% Ta větší mezera je tabulátor. Ve VS Code běžte do 
% File -> Preferences -> Settings -> Commonly Used -> Editor Tab Size
% A nastavte si to na 5, je to pro psaní TeXu trochu přehlednější než defaultní 4. 
% NEPOUŽÍVEJTE MEZERY NA ODSAZENÍ, PRO NAŠE VLASTNÍ DOBRO.
\begin{proof}
	Lemmata jsou dokazována, tedy měli bychom nyní podat důkaz. 
	Připravil jsem tenzorovou abecedu, takže stačí psát
	{\tt
	$\backslash$tenA\{\^{}$\backslash$mu\_;\_$\backslash$nu\} 
	}
	a dostaneme
	\begin{equation}
		\tenA{^\mu_;_\nu}\ .
	\end{equation}
	Máme samozřejmě i malé písmena:
	\begin{equation}
		\tenu{^\mu} \tenu{^\nu} \teng{_\mu_\nu} = -1.
	\end{equation}
	Další zavedené abecedy jsou \texttt{mathbb} a \texttt{mathcal}:
	\begin{center}
		\texttt{$\backslash$bbQ} = $\bbQ$ a \texttt{$\backslash$calQ} = $\calQ$.
	\end{center}
\end{proof}
Konec důkazu je jasně vidět díky končící šedé čáře a díky čtverečku. Později se podíváme i na tvrzeni \ref{claim:cervenanovo_blbe_tvrzeni} a definici \ref{claim:cervenanova_blba_definice}.s
Pokračujeme větou.
\begin{theorem}[Červeňanova blbá věta]
	Pro všechna $a,b\in\mathbb{R}$ platí, že
	\begin{equation}
		(a+b)^2=a^2+b^2 \iff a=0 \vee b=0
	\end{equation}
\end{theorem}
\begin{proof}[Červeňanovy blbé věty]
	Věty jsou dokazovány, tedy měli bychom nyní podat důkaz.
	Všimněte si, že i v důkazu entrujeme věty. 
	Také si všimněte, že tabulátory nekurví počet mezer ani nijak nekurví text, projeví se jen v kódu.
	
	Je dobré si taky všimnout, že tento důkaz je pojmenovaný.
	Konvence je: \texttt{$\backslash$begin\{proof\}[Červeňanovy blbé věty]}.
	Tedy při pojmenovávání v kódu se ptáme: 
	
	\uv{Důkaz jaké které věty?}

	\uv{Důkaz Červeňanovy blbé věty.}
\end{proof}

V další části textu si ukážeme tvrzení.
\begin{claim}[Červeňanovo blbé tvrzení]
	\label{claim:cervenanovo_blbe_tvrzeni}
	Pro všechna $a,b\in\mathbb{R}$ platí, že
	\begin{equation}
		(a+b)^2=a^2+b^2 \iff a=0 \vee b=0
	\end{equation}
\end{claim}
A na závěr definici.
\begin{definition}[Červeňanova nesmyslná definice]
	\label{claim:cervenanova_blba_definice}
	Pro všechna $a,b\in\mathbb{R}$ platí, že
	\begin{equation}
		(a+b)^2=a^2+b^2 \iff a=0 \vee b=0
	\end{equation}
\end{definition}          

\end{document}